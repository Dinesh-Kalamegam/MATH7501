\documentclass[11pt]{article}
\usepackage[left=3cm,right=3cm,top=3cm,bottom=3cm]{geometry}
\usepackage{amsmath}
\usepackage{amsthm}
\usepackage{amssymb}
\usepackage{bm}
\usepackage{tabularx}

\begin{document}
\title{MATH7501: Exercise 8 Solutions}
\author{Dinesh Kalamegam}
\date{\today}
\maketitle

\renewcommand\qedsymbol{\textbf{\emph{Quod Erat Demonstrandum}}}
\setlength{\parindent}{0pt}
\setlength{\parskip}{\baselineskip}
\numberwithin{equation}{subsection}
\newtheorem{theorem}{Theorem}[section]
\newtheorem{definition}[theorem]{Defintion}
\newtheorem{proposition}[theorem]{Proposition}
\newtheorem{corollary}[theorem]{Corollary}

\section{Question 1 (6 MARKS)}
\subsection{Showing $s^{2}$ is a unbiased estimator of $\sigma^{2}$ also states its variance}
Given $s^{2} \sim \Gamma(\alpha,\lambda)$ for $\alpha = \frac{n-1}{2}$ and $\lambda=\frac{n-1}{2\sigma^{2}}$

\begin{align*}
  E(s^{2}) &=  \frac{\alpha}{\lambda} \\
           &=  \frac{(n-1)/2}{(n-1)/2 \sigma^{2}} \\
           &=  \frac{2 \sigma^{2}}{2} \\
           &=  \boxed{\sigma^{2}}
\end{align*}
Then the $Bias(s^{2})=E(s^{2})-\sigma^{2}=0$ means that $s^{2}$ is an unbiased estimator for $\sigma^{2}$

\begin{align*}
  Var(s^{2}) &= \frac{\alpha}{\lambda^{2}} \\
             &= \frac{(n-1)/2}{(n-1)^{2})/4 \sigma^{2}} \\
             &= \boxed{\frac{2 \sigma^{4}}{n-1}}
\end{align*}
\subsection{}
\subsubsection{Find expressions for the bias and variance of $T_{k}$ as an estimator of $\sigma^{2}$}
Consider $T_{k} = ks^{2}$.
\begin{align*}
  E(T_{k})  &= E(ks^{2}) = kE(s^{2}) = k\sigma^{2} \\
  Var(T_{k}) &= Var(ks^{2}) = k^{2}Var(s^{2}) = \boxed{\frac{2k^{2}\sigma^{4}}{n-1}}
\end{align*}
$Bias(T_{k}) = E(T_{k}) - \sigma^{2}$ as an estimator of $\sigma^{2}$ which then gives us $k\sigma^{2}-\sigma^{2} = \boxed{\sigma^{2}(k-1)}$

Then the Mean Square Error (MSE) of $T_{k}$ is
\begin{align*}
  MSE(T_{k}) &= Bias^{2}(T_{k}) + Var(T_{k}) \\
             &= (k-1)^{2}\sigma^{2} + \frac{2k^{2}\sigma^{4}}{n-1}
\end{align*}

\subsection{Minimise the error}
To minimise the MSE in terms of $k$ consider
\begin{align*}
  \frac{d}{dk} MSE(T_{k}) = 2\sigma^{4}(k-1) + \frac{4\sigma^{4}}{n-1}k
\end{align*}
Then by setting this derivate to zero and assuming $\sigma \neq 0$ we obtain
\begin{align*}
  2\sigma^{4}(k-1)+4\sigma^{4}k =0 \\
  & = 2\sigma^{4} \left ((k-1)+\frac{2k}{n-1} \right) = 0 \\
  &\implies \boxed{k = \frac{n-1}{n+1}}
\end{align*}
To confirm that this leads to a minimum value of the MSE consider the second derivative
\begin{align*}
  \frac{\partial^{2}}{\partial k} MSE(T_{k}) &= 2\sigma^{4}+\frac{4\sigma^{4}}{n-1}>0
\end{align*}
So we have a minimum
\section{Question 2 (4 MARKS)}
\subsection{ }
$Z \sim N(0,1)$ and $U \sim \chi_{\nu}^{2}$ So we have that $Z^{2} \sim \chi_{1}^{2}$. Then by defintion

\begin{align*}
  T &= \frac{Z}{\sqrt{\frac{U}{\nu}}} \\ \\
  & \boxed{\text{Is a $t_{\nu}$ distribution with $\nu$ degrees of freedom}}
\end{align*}

Then for the next Part
\begin{align*}
  T^{2} \sim  \frac{Z^{2}}{U/\nu} &=\frac{Z^{2}/1}{U/\nu} \\ \\
        &= \frac{\chi_{1}^{2}}{\chi_{\nu}^{2}} \\ \\
        &= \boxed{\text{distributed by } F_{1,\nu}}
\end{align*}
\subsection{}
\subsubsection{$P(Y>4)$}
Given that $Y \sim F_{1,5}$ we can use 2.1 to give $Y=T^{2}$ and $T \sim t_{5}$.

The required probability is then
\begin{align*}
  P(Y>4) &= P(T^{2}>4) \\
         &= P(T>2) + P(T<-2) \\
         &= 2P(T>2) \text{ by symmetry of the $t$ distribution} \\
         &= 2(1-P(T<2)) \\
         &= 2(1-0.9490)\\
         &= \boxed{0.102}
\end{align*}
\subsubsection{find $c$ s.t. $P(Y>c)=0.01$ }
This follows the steps above and this leads us to
\begin{align*}
  P(Y>c) &= P(T^{2}>c) \\
         &= P(T>\sqrt{c}) + P(T<-sqrt{c}) \\
         &= 2P(T>\sqrt{c}) \text{ by symmetry of the $t$ distribution} \\
         &= 2(1-P(T<\sqrt{c})) = 0.01\\
         &\implies 1- P(T<\sqrt{c}) = 0.005 = 0.5\% \\
         & \text{Recall that $T \sim t_{5}$ so from table 10 read from row $\nu=5$ and column $0.5$} \\
         &\implies \sqrt{c} = 4.032 \\
         &\implies \boxed{c=16.257 \text{ to 3sf} }
\end{align*}
\section{Question 3 (10 MARKS)}
\subsection{ }
\subsubsection{Stem and Leaf diagram}
%\begin{tabularx}{\textwidth}{ X|X }
%  $k$ & $0$ \\
%  \hline
%  $P(X=k$) & $1/4$ \\
%  \hline

\begin{verbatim}
Key 197| 8 = 197.8 cm

  197 | 8  9  9
  198 | 0  2  3
  199 | 2  6  7
  200 | 2  2  5
  201 | 4  8
  202 | 0
  203 | 3
\end{verbatim}
\subsubsection{Sample mean and variance}
For this data we have $n=16$ and
\begin{align*}
  \sum_{i=1}^{16} X_{i} = 3196 \\
  \sum_{i=1}^{n} X_{i}^{2} = 638445.1
\end{align*}
Then the sample mean
\begin{align*}
  \bar{X} &= \frac{3196}{16} \\
          &= \boxed{197.75}
\end{align*}
And the sample variance
\begin{align*}
  s^{2} &= \frac{\sum_{i=1}^{16}X_{i}^{2}-n\bar{X}}{n-1} \\
        &= \frac{1}{15}(638445.1 -16(199.75^{2})) \\
        &= \boxed{2.94}
\end{align*}
\subsection{95\% confidence interval}
From the notes we have that $\bar{X} \sim N(\mu,\frac{\sigma^{2}}{n})$ i.e $Z = \frac{\bar{X}-\mu}{\sigma/\sqrt{n}}$

At a $95\%$ confidence interval for $Z$ is $(-1.96,1.96)$ and from the tables we have that

$P(Z<1.96)= 0.975 \land P(Z>1.96) = 0.025$

Then the confidence interval for $\mu$ can be found using $-1.96<\frac{\bar{X}-\mu}{\sigma/\sqrt{n}} <1.96$ we also have $\bar{X}$ as well as the fact that question gives that $\sigma=1$
\begin{align*}
  & \frac{\bar{X}-1.96}{\sigma/\sqrt{n}} < \mu < \frac{\bar{X}+1.96}{\sigma/\sqrt{n}} \\ \\
  &\frac{199.75-1.96}{1/4} < \mu < \frac{199.75+1.96}{1/4} \\ \\
  & 199.26 < \mu < 200.24
\end{align*}
This gives us the confidence interval $\boxed{(199.26,200.24)\text{cm}}$ The interval includes the nominal value of $200$ cm and hence data is consistent with underlying mean of $200$cm

\end{document}
